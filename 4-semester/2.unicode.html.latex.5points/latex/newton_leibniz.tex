\documentclass{article}
\usepackage[russian]{babel}
\usepackage{amsmath}
\usepackage{amssymb}

\title{Доказательство формулы Ньютона-Лейбница}
\author{}
\date{}

\begin{document}

\maketitle

\section*{Формула Ньютона-Лейбница}

Формула Ньютона-Лейбница утверждает, что если \( f \) является непрерывной функцией на отрезке \([a, b]\) и \( F \) является её первообразной, то
\[
\int_{a}^{b} f(x) \, dx = F(b) - F(a).
\]

\section*{Доказательство}

Пусть \( f \) — непрерывная функция на \([a, b]\), и пусть \( F \) — первообразная \( f \), то есть \( F'(x) = f(x) \) для всех \( x \in [a, b] \).

По определению определённого интеграла:
\[
\int_{a}^{b} f(x) \, dx = \lim_{n \to \infty} \sum_{i=1}^{n} f(x_i^*) \Delta x_i,
\]
где \( \{[x_{i-1}, x_i]\} \) — разбиение отрезка \([a, b]\) на \( n \) равных частей, \( \Delta x_i = \frac{b-a}{n} \), и \( x_i^* \) — произвольная точка в интервале \( [x_{i-1}, x_i] \).

Теперь рассмотрим разность \( F(x_i) - F(x_{i-1}) \). По теореме о среднем значении для интегралов существует такая точка \( c_i \in [x_{i-1}, x_i] \), что:
\[
F(x_i) - F(x_{i-1}) = F'(c_i) (x_i - x_{i-1}).
\]
Так как \( F'(x) = f(x) \), получаем:
\[
F(x_i) - F(x_{i-1}) = f(c_i) \Delta x_i.
\]

Суммируя по всем интервалам разбиения, получаем:
\[
\sum_{i=1}^{n} (F(x_i) - F(x_{i-1})) = \sum_{i=1}^{n} f(c_i) \Delta x_i.
\]

Левая часть этой суммы — это телескопическая сумма, которая сворачивается в:
\[
F(b) - F(a).
\]

Таким образом, имеем:
\[
F(b) - F(a) = \sum_{i=1}^{n} f(c_i) \Delta x_i.
\]

Переходя к пределу при \( n \to \infty \), мы видим, что правая часть выражения стремится к определённому интегралу:
\[
\int_{a}^{b} f(x) \, dx = F(b) - F(a).
\]

Таким образом, формула Ньютона-Лейбница доказана:
\[
\int_{a}^{b} f(x) \, dx = F(b) - F(a).
\]

\end{document}
